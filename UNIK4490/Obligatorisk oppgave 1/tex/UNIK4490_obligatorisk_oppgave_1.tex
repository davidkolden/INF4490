\documentclass[norsk]{article}
\usepackage[norsk]{babel}
\usepackage{parskip}
\usepackage[utf8]{inputenc}
\usepackage{graphicx}
\usepackage{float}

\renewcommand{\thesubsection}{\thesection.\alph{subsection}}
\author{David Kolden, davidko}

\title{UNIK 4490 - Obligatorisk oppgave 1}
\begin{document}
\maketitle
\section{Øvelse 1}

\subsection{ }
Finner poler ved å løse \(s(1 + T_Ms) = 0\) som gir polene \(s = 0\) og \(s = -\frac{1}{T_M}\). Systemet er stabilt for alle positive verdier av \(T_M\).

\subsection{ }
Figur 1 viser blokkskjema for \(\frac{X(s)}{U(s)} = H(s) = \frac{1}{s(1 + T_Ms)}\)
\begin{figure}[!htb]
\includegraphics[height=3.5cm]{illustrations/oppg1b_illu}
\caption{Blokkskjema for \(H(s)\)}
\end{figure}

\subsection{ }
\(H(s)\) har to poler og er derfor et andreordens system.

Setter \(U(s) = K(1 + T_Ds)E(s)\), \(E(s) = R(s) - X(s)\), \(U(s) = K(1 + T_Ds)(R(s) - X(s))\) sammen med \(H(s)\):
\[X(s) = H(s)U(s) = H(s)K(1+T_Ds)(R(s) - X(s))\] 
\[X(s) = H(s)K(1+T_Ds)R(s) - H(s)K(1+T_Ds)X(s)\]
\[X(s)(1 + H(s)K(1+T_Ds)) = H(s)K(1+T_Ds)R(s)\]
\[\frac{X(s)}{R(s)} = H_C(s) = \frac{H(s)K(1+T_Ds)}{1+H(s)K(1+T_Ds)}\]
\[H_C(s) = \frac{K(1+T_Ds)}{\frac{1}{H(s)} + K(1+T_Ds)}\]

Setter inn for \(H(s)\):
\[H_C(s) = \frac{K(1+T_Ds)}{s(1+T_Ms)+ K(1+T_Ds)}\]
\[H_C(s) = \frac{K(1+T_Ds)}{s^2T_M + s + KT_Ds + K}\]
\[H_C(s) = \frac{(1+T_Ds)}{s^2\frac{T_M}{K} + s(\frac{1}{K}+T_D) + 1}\]

Ser at systemet med kontroller fortsatt er et andreordens system.
\subsection{ }
Figur to viser blokkskjema for systemet med kontroller (\(H_C(s)\))
\begin{figure}[!htb]
\includegraphics[height=2.9cm]{illustrations/oppg1d_illu}
\caption{Blokkskjema for \(H_C(s)\)}
\end{figure}

\subsection{ }
\(H_C(s)\) har ett nullpunkt og to poler. Nullpunktet finnes ved å sette telleren i \(H_C(s)\) til null, mens man finner polene ved å sette nevneren til null. Polene kan dermed finnes med uttrykket
\[s = \frac{-(\frac{1}{K} + T_D) \pm \sqrt{(\frac{1}{K} + T_D)^2 - 4\frac{T_M}{K}}}{2\frac{T_M}{K}}\]

mens nullpunktene finnes med uttrykket
\[s = -\frac{1}{T_D}\]

Ved å sette inn for \(T_M = 2\) og \(T_D = 1\) får vi til slutt et nullpunkt i \(s = -1\) og to poler i
\[s = \frac{-(\frac{1}{K} + 1) \pm \sqrt{(\frac{1}{K} + 1)^2 - 4\frac{2}{K}}}{2\frac{2}{K}}\]

\begin{figure}[!htb]
\includegraphics[height=10cm]{illustrations/oppg1e_illu}
\caption{Locusplot av \(H_C\)}
\end{figure}

Med \(K \approx 0.17\), så er systemet 

\section{Øvelse 2}
Et system kan verifiseres som stabilt for en kandidatfunksjon \(V(x,y)\) hvis
\begin{itemize}
\item \(V(x, y) > 0 \qquad \forall x \neq 0, y \neq 0\)
\item \(V(x,y) = 0 \qquad x = y = 0\)
\item \(V(x,y) \rightarrow \infty \qquad x \rightarrow \infty, y \rightarrow \infty\)
\item \(\dot{V}(x,y) < 0 \)
\end{itemize}
Med en kandidatfunksjon \[V(x,y) = x^2 +  y^2\] ser vi at kravene fra første, andre og tredje punkt er godkjente ettersom begge uttrykkene er kvadratiske.

Med systemet \[\dot{x} = -y-x^3\] \[\dot{y} = x - y^3\] kan systemet verifiseres ved å finne \(\dot{V}(x, y)\).

\[\dot{V}(x,y) = 2x\dot{x} + 2y\dot{y}\]
\[\dot{V}(x,y) = 2x(-y-x^3) + 2y(x-y^3)\]
\[\dot{V}(x,y)=-2xy-2x^4 + 2xy-2y^4\]
\[\dot{V}(x,y) = -x^4-y^4\]

Vi ser at \(\dot{V}(x,y)\) er godkjent i forhold til det siste kravet ettersom \(x^4\) og \(y^4\) ikke kan bli negative.



\section{Øvelse 3}
\subsection{ }
\subsection{ }
\subsection{ }
\subsection{ }
\subsection{ }
\subsection{ }
\subsection{ }
\subsection{ }
\end{document}