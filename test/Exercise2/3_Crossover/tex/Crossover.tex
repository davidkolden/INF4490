\documentclass[12pt]{article}
\author{David Kolden}
\title{Crossover methods}
\date{\today}
\usepackage{graphicx}
\usepackage{float}
\begin{document}
\maketitle
\tableofcontents
\newpage
\section{Crossover algorithms}
\subsection{Partially mapped crossover[PMX]}
The partially mapped crossover algorithm is summarized in \textit{figure 1}:\\
\begin{figure}[H]
\begin{center}
\includegraphics[scale=0.7]{"../fig/Partially mapped crossover".jpg}
\caption{Flow of the PMX algorithm.}
\end{center}
\end{figure}
\newpage
\subsection{Order crossover}
The order crossover algorithm is summarized in \textit{figure 2}:\\
\begin{figure}[H]
\begin{center}
\includegraphics[scale=0.7]{"../fig/Order crossover".jpg}
\caption{Flow of the order crossover algorithm.}
\end{center}
\end{figure}
\newpage
\subsection{Cycle crossover}
The cycle crossover algorithm is summarized in \textit{figure 3}:\\
\begin{figure}[H]
\begin{center}
\includegraphics[scale=0.5]{"../fig/Cycle crossover".jpg}
\caption{Flow of the cycle crossover algorithm.}
\end{center}
\end{figure}
\newpage
\section{Strategy}
There seems to be some similarities between PMX and order crossover. Some code can be reused. These algorithms will be implemented as member functions of class Crossover.
\end{document}